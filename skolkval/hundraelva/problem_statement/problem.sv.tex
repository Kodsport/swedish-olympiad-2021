\problemname{Hundraelva kronor}
I Tumba pappersbruk --- som är ansvariga för att producera sedlar --- har tryckpressen gått sönder: den kan nu bara trycka siffran ”\texttt{1}”.
Att köpa en ny tryckpress kostar $N$ kronor men pappersbruket har tyvärr helt slut på pengar.
Det är ju dock de själva som trycker sedlar, så varför inte trycka nya pengar så att de kan köpa den nya maskinen?

Eftersom den trasiga tryckpressen bara kan trycka siffran ”\texttt{1}” kan de endast trycka sedlar med valörerna $1$ krona, $11$ kronor, $111$ kronor, $1111$ kronor, o.s.v.

Pappersbruket undrar nu hur många sedlar de behöver trycka för att kunna betala för den nya tryckpressen.
De vill kunna betala med jämna pengar, d.v.s. exakt $N$ kronor (det är omoraliskt att trycka upp mer pengar än de behöver), och därför vill de trycka så få sedlar som möjligt.
Skriv ett program som beräknar antalet sedlar de måste trycka.

\section*{Indata}
Den första och enda raden innehåller ett heltal $N$ ($1 \le N \le 10^9)$, kostnaden i kronor för den nya tryckpressen.

\section*{Utdata}
Skriv ut ett heltal: det minsta antalet sedlar som behöver tryckas.

\section*{Poängsättning}
Din lösning kommer att testas på en mängd testfallsgrupper.
För att få poäng för en grupp så måste du klara alla testfall i gruppen.

\noindent
\begin{tabular}{| l | l | l |}
  \hline
  \textbf{Grupp} & \textbf{Poängvärde} & \textbf{Gränser} \\ \hline
  $1$    & $20$        &  $ N \leq 1000 $ \\ \hline 
  $2$    & $40$        &  $ N \leq 10^6 $ \\ \hline
  $3$    & $40$        &  Inga ytterligare begränsningar. \\ \hline
\end{tabular}

\section*{Förklaring av exempel}
\begin{itemize}
  \item I det första exempelfallet kan man använda en $1$-kronasedel och två $11$-kronorssedlar.
  \item I det andra exempelfallet kan man använda en av varje av $1$-, $11$-, $111$-, $1111$-, $11111$-kronorssedel.
\end{itemize}
