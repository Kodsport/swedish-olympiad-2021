\problemname{One Hundred and Eleven Kronor}

At Tumba paper mill --- which is responsible for producing banknotes --- the printing press has broken down:
it can now only print the digit ”\texttt{1}”.
Buying a new printing press costs $N$ Swedish kronor, but unfortunately, the paper mill has completely run out of money.
However, since they themselves print banknotes, why not print new money so they can buy the new machine?

Because the broken printing press can only print the digit ”\texttt{1}”, they can only print banknotes with
the denominations of $1$ krona, $11$ kronor, $111$ kronor, $1111$ kronor, and so on.

The paper mill now wonders how many banknotes they need to print to be able to pay for the new printing press.
They want to be able to pay with exact money, i.e., exactly $N$ kronor (it is immoral to print more money than they need),
and therefore they want to print as few banknotes as possible.
Write a program that calculates the number of banknotes they must print.

\section*{Input}
The first and only line contains an integer $N$ ($1 \leq N \leq 10^9$), the cost in kronor for the new printing press.

\section*{Output}
Print an integer: the minimum number of banknotes that need to be printed.

\section*{Points}
Your solution will be tested on several test case groups.
To get the points for a group, it must pass all the test cases in the group.

\noindent
\begin{tabular}{| l | l | p{12cm} |}
  \hline
  \textbf{Group} & \textbf{Point value} & \textbf{Constraints} \\ \hline
  $1$    & $20$        &  $ N \leq 1000 $ \\ \hline 
  $2$    & $40$        &  $ N \leq 10^6 $ \\ \hline
  $3$    & $40$        &  No additional constraints. \\ \hline
\end{tabular}



\section*{Explanation of samples}
\begin{itemize}
  \item In the first example case, one can use one $1$-krona banknote and two $11$-kronor banknotes.
  \item In the second example case, one can use one of each $1$-, $11$-, $111$-, $1111$-, and $11111$-krona banknote.
\end{itemize}
