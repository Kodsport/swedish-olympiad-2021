\problemname{Äpplen och päron}
Axel vill tävla med Petra om vem som sålt flest äpplen respektive päron, men Petra tycker inte att man kan ``jämföra äpplen och päron''. De kommer överens om att de istället ska jämföra hur mycket de tjänat. De ber dig skapa ett program som, givet antalet äpplen Axel sålt och antalet päron Petra sålt, skriver ut vem som tjänat mest (eller ``\texttt{lika}'' om de sålt för lika mycket). Äpplen kostar 7 kronor styck och päron 13 kronor styck.

\begin{center}
  \includegraphics[width=5cm]{apple.jpg}
  \includegraphics[width=5cm]{pear.jpg}
\end{center}

\section*{Indata}
En rad med två heltal $A,P$ ($0 \le A,P \le 1000)$, antalet äpplen Axel har lyckats sälja, och antalet päron Petra har lyckats sälja. 

\section*{Utdata}
Programmet ska skriva ut en rad med en sträng: den person som har tjänat mest, ``\texttt{Axel}'' eller ``\texttt{Petra}''. Om de sålt för lika mycket ska du skriva ``\texttt{lika}''.

\section*{Poängsättning}
Din lösning kommer att testas på en mängd testfallsgrupper.
För att få poäng för en grupp så måste du klara alla testfall i gruppen.

\noindent
\begin{tabular}{| l | l | l |}
  \hline
  Grupp & Poängvärde & Gränser \\ \hline
  $1$   & $40$       & Axel och Petra har inte tjänat lika mycket pengar. \\ \hline
  $2$   & $60$       & Inga ytterligare begränsningar. \\ \hline
\end{tabular}