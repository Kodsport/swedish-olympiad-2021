\problemname{Personal Identification Number}

Swedish personal identification numbers are usually written with ten digits in the format \texttt{YYMMDD-XXXX}.
The first six digits constitute the person's birth date, so a person with the identification number \texttt{781113-3285}
was, for example, born on November 13, 1978. A detail that many people do not know is that people over 100
years old get a plus sign instead of a hyphen in their identification number. This is so that they are not
confused with people born exactly 100 years later. For example, someone born in 1912 can have the identification number
\texttt{121212+1212}, while \texttt{121212-1212} belongs to someone born in 2012.

Write a program that reads an identification number in the format above,
and prints it in \emph{twelve-digit} format, i.e., \texttt{YYYYMMDDXXXX}.
You can assume that the people in the input were born between 1840 and 2019.
For simplicity, there will also be no people born in 1920.


\section*{Input}
The first and only line contains a string consisting of digits, hyphens, and plus signs, in the format described above.

\section*{Output}
Print a string, the personal identification number from the input converted to twelve-digit format.

\section*{Points}
Your solution will be tested on several test case groups.
To get the points for a group, it must pass all the test cases in the group.

\noindent
\begin{tabular}{| l | l | p{12cm} |}
  \hline
  \textbf{Group} & \textbf{Point value} & \textbf{Constraints} \\ \hline
  $1$    & $50$      & The personal identification number never contains a plus-sign. \\ \hline
  $2$    & $50$      & No additional constraints. \\ \hline
\end{tabular}



%\section*{Samples}
