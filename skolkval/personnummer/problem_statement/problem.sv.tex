\problemname{Personnummer}

Svenska personnummer skrivs oftast med tio siffror på formatet \texttt{ÅÅMMDD-XXXX}. De sex första siffrorna utgör personens födelsedatum, så en
person med personnummer \texttt{781113-3285} är t.ex.\ född den $13$:e november $1978$. En detalj som många
inte känner till är att hundraplussare får ett plustecken istället för bindestreck i sitt 
personnummer. Detta är för att man inte ska blanda ihop dem med personer som föddes exakt $100$
år senare. Till exempel kan någon som föddes år $1912$ ha personnummret \texttt{121212+1212}, medan 
\texttt{121212-1212} tillhör någon som föddes $2012$. 

Skriv ett program som läser in ett personnummer på formatet ovan,
och skriver ut det på \emph{tolvsiffrigt} format, d.v.s.\ \texttt{ÅÅÅÅMMDDXXXX}.
Du kan anta att personerna i indatan föddes mellan $1840$ och $2019$.
För enkelhets skull kommer det inte heller finnas några personer som föddes $1920$. 

\section*{Indata}
Den första och enda raden innehåller en sträng bestående av siffror, bindestreck och plustecken, på formatet ovan.

\section*{Utdata}
Skriv ut en sträng, personnumret i indatan omvandlat till tolvsiffrigt format.

\section*{Poängsättning}
Din lösning kommer att testas på en mängd testfallsgrupper.
För att få poäng för en grupp så måste du klara alla testfall i gruppen.

\noindent
\begin{tabular}{| l | l | l |}
  \hline
  Grupp & Poängvärde & Gränser \\ \hline
  $1$   & $50$       & Personnummret innehåller aldrig ett plustecken. \\ \hline
  $2$   & $50$       & Inga ytterligare begränsningar. \\ \hline
\end{tabular}
