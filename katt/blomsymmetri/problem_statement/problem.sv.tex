\problemname{Blomsymmetri}

Fredrik har just ställt ut nya blommor på sin fönsterbräda. Men någonting känns lite fel. 
Det är inte riktigt så symmetriskt som han skulle vilja ha det.

Det finns $K$ olika blomsorter och varje sort beskrivs av ett av heltalen $1,2,...,K$. Fönsterbrädan har $N$ blommor utställda på rad från vänster till höger.
Den $i$:te blomman från vänster är av sort $a_i$.
Fredrik vill att blommorna ska vara symmetriska, d.v.s att $a_i=a_{N+1-i}$ gäller för alla $i$. 
För att åstakomma detta tänker han göra drag som byter plats på två intilliggande blommor.
Ett drag kan alltså byta plats på blomma $i$ och blomma $i+1$ för något $1\le i \le N-1$.

Vad är minsta antalet drag som krävs för att uppnå blomsymmetri?
\section*{Indata}

Den första raden av indata innehåller två heltal $N$ och $K$ ($1 \le N,K \le 200 000$): 
antalet blomor på fönsterbrädan och totala antalet sorter som finns.
Därefter följer en rad med $N$ heltal $a_1,a_2,...,a_N$ ($1\le a_i \le K$): blommorna på fönsterbrädan.

Det garanteras att det finns en sekvens av drag som ordnar blommorna symmetriskt.

\section*{Utdata}

Skriv ut ett heltal: det minsta möjliga antalet drag för att uppnå blomsymmetri.

\section*{Poängsättning}
Din lösning kommer att testas på en mängd testfallsgrupper.
För att få poäng för en grupp så måste du klara alla testfall i gruppen.

\noindent
\begin{tabular}{| l | l | l |}
  \hline
  Grupp & Poängvärde & Gränser \\ \hline
  $1$    & $9$         &  $N\le 10, K \le 10$ \\ \hline
  $2$    & $11$        &  $N \le 20$ \\ \hline
  $3$    & $13$        &  $N \le 2000$ \\ \hline
  $4$    & $17$        &  $K \le 2$ \\ \hline
  $5$    & $20$        &  $N$ är jämn \\ \hline 
  $6$    & $30$        &  Inga ytterligare begränsningar \\ \hline
\end{tabular}

\section*{Förklaring av exempel}
I det första exemplet kan vi börja med att byta plats på blomma 2 och 3, och sedan på blomma 1 och 2.
Då kommer blommorna hamna i ordningen \texttt{1 3 3 1}.

I det andra exemplet kan vi till exempel göra följande 5 byten i ordning:
4 och 5, 3 och 4, 1 och 2, 2 och 3, 7 och 8.
Då kommer blommorna hamna i ordningen \texttt{1 2 3 1 1 3 2 1}.

