\problemname{Tramway}
\noindent
The public transportation system in Stackköping consists of a \textit{tramway} using $N$ stations and $M$ lines.
Each line consists of an ordered sequence of stations where it is possible to travel in both directions.
A \textit{journey} on the tramway is a movement between two stations on the same line.
For example, if the line consists of the stations $(3,7,5,2,1,9)$, a journey from $1$ to $5$
will pass through stations $1$, $2$, and $5$.

A common health issue in Stackköping is that people take short trips on the tramway instead
of walking. To counteract this, the municipality has decided to switch to a new payment system
where the price for a journey is proportional to the \textit{waste}. The waste of a journey
is defined as the number of stations on the line that the tramway does not pass through. In the example above,
the waste is $3$, because stations $3,7,$ and $9$ are not passed through.

You want to travel from station $1$ to station $N$ through a sequence of journeys.
What is the minimum possible total waste?
It is guaranteed that it is possible to reach all stations from station $1$.

\section*{Input}
The first line of input contains two integers $N$ and $L$ ($2 \leq N,L \leq 10^5$),
the number of stations and the number of lines in the tramway network.

The following $L$ lines each describe a tramway line. Each of these lines begin with an
integer $M$ ($2 \le M \le N$) followed by $M$ integers between $1$ and $N$, the number of
stations on the tramway line and its stations. These $M$ integers are all distinct.

Let $S$ be the sum of $M$ over all tramway lines. It is guaranteed that $S \leq 3 \cdot 10^5$.

\section*{Output}
Print an integer: the smallest possible total waste required to travel from station $1$ to
station $N$.

\section*{Points}
Your solution will be tested on several test case groups.
To get the points for a group, it must pass all the test cases in the group.

\noindent
\begin{tabular}{| l | l | p{12cm} |}
  \hline
  \textbf{Group} & \textbf{Point value} & \textbf{Constraints} \\ \hline
  $1$    & $10$        &  $N,\le 100, S \le 200$ \\ \hline
  $2$    & $10$        &  $N,\le 2000, S \le 4000$ \\ \hline
  $3$    & $10$        &  $M \le 100, S \le 30000$ \\ \hline
  $4$    & $15$        &  $L=1$ \\ \hline 
  $5$    & $55$        &  No further constraints. \\ \hline
\end{tabular}


\section*{Explanation of Samples}
In the first sample, we first travel from $1$ to $2$ using the first line. We can then travel from
$2$ to $6$ using the second line. The total waste for the trips is $1$ and $2$, therefore the answer is $3$.

In the second sample we can start by travelling from $1$ to $4$, with waste $1$. Afterwards, we travel from
$4$ to $5$, which has a waste of $0$ as all stations along the line are visited.
