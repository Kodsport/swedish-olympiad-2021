\problemname{Bybana}

Kollektivtrafiken i Stackköping utgörs av en \textit{bybana} med $N$ stationer och $M$ linjer.
Varje linje består av en ordnad sekvens av stationer där det går att åka i båda riktningarna.
En $\textit{resa}$ på bybanan är en förflyttning mellan två stationer på en och samma linje. 
Om linjen exempelvis består av stationerna $(3,7,5,2,1,9)$ så kommer en resa från $1$ till $5$
att åka förbi stationerna $1$, $2$, $5$.

Ett vanligt hälsoproblem i Stackköping är att folk åker korta sträckor på bybanan istället
för att gå. För att motverka detta har kommunen beslutat att byta till ett nytt betalsystem 
där priset för en resa är proportionellt mot \textit{slöseriet}. Slöseriet för en resa 
definieras som antalet stationer på linjen som bybanan inte åkte förbi. I exemplet ovan så 
är slöseriet $3$, eftersom stationerna $3,7,9$ inte åktes förbi.

Du vill ta dig från station $1$ till station $N$ genom en sekvens av resor.
Vad är det minsta möjliga totala slöseriet?
Det är garanterat att det går att ta sig till alla stationer.

\section*{Indata}

Den första raden av indata innehåller två heltal $N$ och $L$ ($2 \le N,L \le 10^5$): 
antalet stationer respektive antalet linjer i bybanenätverket.
Därefter följer $L$ rader som var och en beskriver en linje. En linje beskrivs
av ett tal $M$ ($2 \le M \le N$) följt av $M$ tal mellan $1$ och $N$: antalet stationer på
linjen respektive stationerna på linjen. Dessa $M$ tal är garanterat distinkta.

Låt $S$ vara summan av $M$ över alla linjer. Det garanteras att $S \le 3\cdot 10^5$.

\section*{Utdata}

Skriv ut ett tal: det minsta möjliga totala slöseriet för en färd från station $1$ till
station $N$.

\section*{Poängsättning}
Din lösning kommer att testas på en mängd testfallsgrupper.
För att få poäng för en grupp så måste du klara alla testfall i gruppen.

\noindent
\begin{tabular}{| l | l | l |}
  \hline
  Grupp & Poängvärde & Gränser \\ \hline
  $1$    & $10$        &  $N,\le 100, S \le 200$ \\ \hline
  $2$    & $10$        &  $N,\le 2000, S \le 4000$ \\ \hline
  $3$    & $10$        &  $M \le 100, S \le 30000$ \\ \hline
  $4$    & $15$        &  $L=1$ \\ \hline 
  $5$    & $55$        &  Inga ytterligare begränsningar \\ \hline
\end{tabular}

\section*{Förklaring av exempel}
I det första exemplet kan vi först resa från $1$ till $2$ på den första linjen. Sedan kan vi resa
från $2$ till $6$ på den andra linjen. Slöseriet för resorna är $1$ respektive $2$, så svaret är $3$.

I det andra exemplet kan vi börja med att resa från $1$ till $4$, med slöseriet $1$. Därefter kan vi resa
från $4$ till $5$ vilket bidrar med $0$ slöseri eftersom alla stationer besöks på sträckan.